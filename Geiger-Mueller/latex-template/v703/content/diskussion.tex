\section{Diskussion}
\label{sec:Diskussion}

Die Charakteristik \autoref{fig:charakteristik} zeigt nur in geringer Ausprägung den erwarteten 
Kurvenverlauf.\\
Das Plateau ist aufgrund einer Häufung von ausreißenden Werten in jenem Bereich
nicht direkt zu erkennen. Die Steigung der Ausgleichsgerade ergibt sich allerdings wie gewünscht zu einem 
sehr niedrigen Wert von $a = 0,73 \mathrm{\frac{\%}{100 V}}$.\\


Die via Zwei-Quellen-Methode bestimmte Totzeit $T = (253,8 \pm 0,001) \mu$s weicht um 
$\eta_T = (69,2 \pm 0,0007) \% $ vom auf dem Oszilloskop abgelesenen Wert von $T = 150 \mu$s 
ab. 
Beide liegen jedoch im typischen Bereich weniger Hundert Mikrosekunden für die Totzeit eines Geiger-Müller-Zählrohrs.\\
Die Abweichung kann darauf zurückgeführt werden, dass die Oszilloskopmethode
beispielsweise durch Ablesefehler sehr ungenau ist.\\

Die Anzahl der je einfallendem Teilchen freigesetzter Ladungsträger 
ist proportional zur Zählrohrspannung. Dies ist in \autoref{fig:lad} zu sehen und entspricht den
theoretischen Erwartungen.\\