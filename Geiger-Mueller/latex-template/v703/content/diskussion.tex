\section{Diskussion}
\label{sec:Diskussion}

Die Charakteristik \autoref{fig:charakteristik} zeigt nur in geringer Ausprägung den erwarteten 
Kurvenverlauf.\\
Das Plateau ist aufgrund einer Häufung von ausreißenden Werten in jenem Bereich
nicht direkt zu erkennen. Die Steigung der Ausgleichsgerade ergibt sich allerdings wie gewünscht zu einem 
sehr niedrigen Wert von $m = (0.73 \pm 0.45) \mathrm{\frac{\%}{100 V}}$.\\
Die via Zwei-Quellen-Methode bestimmte Totzeit $T = (50 \pm 40) \mu$s weicht um 
$\eta_T = (62 \pm 31) \% $ vom auf dem Oszilloskop abgelesenen Wert von $T = 130 \mu$s 
ab. 
Beide liegen jedoch im typischen Bereich weniger Hundert Mikrosekunden für die Totzeit eines Geiger-Müller-Zählrohrs.\\
Die Abweichung kann darauf zurückgeführt werden, dass die Oszilloskopmethode
beispielsweise durch Ablesefehler sehr ungenau ist.\\
Die Anzahl der je einfallendem Teilchen freigesetzter Ladungsträger 
ist proportional zur angelegten Spannung. Dies ist in \autoref{fig:lad} zu sehen und entspricht den theoretischen Erwartungen.\\
Die Bestimmung der Nachentladungszeit ist sehr ungenau, weil sich die Peaks in \autoref{fig:totzeit} häufig überlappen. Allerdings sind sie trotzdem klar erkennbar in \autoref{fig:charakteristik}, da die Plateaugerade ansteigt.\\
Nach allen Teilen dieses Versuches ist die Funktionsweise des Geiger-Müller-Zählrohrs klar erkennbar.
