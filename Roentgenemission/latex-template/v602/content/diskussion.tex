\section{Diskussion}
Bei den K-Linien weichen die Energien um
\begin{equation*}
  E_{\alpha_{\%}}=2\ \%
\end{equation*}
\begin{equation*}
  E_{\beta_{\%}}=1\ \%
\end{equation*}
und die Winkel um
\begin{equation*}
  \theta_{\alpha_{\%}}=2\ \%
\end{equation*}
\begin{equation*}
  \theta_{\beta_{\%}}=2\ \%
\end{equation*}
ab. Das ist wieder in einem geringen Rahmen und akzeptabel.

\subsection{Überprüfung der Bragg-Bedingung}
Wird der Theoriewert mit dem experimentellen Wert verglichen, ergibt sich zu
\begin{equation*}
  |\frac{\theta_{theo}-\theta_{exp}}{\theta_{exp}}|=0.08\%,
\end{equation*}
woraus sich erkennen lässt, dass die Abweichung sehr gering ist und die Bragg-Bedingung erfüllt.

\subsection{Emissionsspektrum}

\subsection{Absorptionsspektren}
Für die Energien, Winkel und Abschirmkonstanten ergeben sich folgende Abweichungen:
\begin{table}[H]
  \centering
  \begin{tabular}{l|l|l|l}
  & $E_{\%}$ & $\theta_{\%}$ & $\sigma_{\%}$\\ \hline
  Zn & $5.2\ \%$ & $5\ \%$ & $18.8\ \%$\\ \hline
  Ga & $10.6\ \%$ & $10.5\ \%$ & $41.6\ \%$\\ \hline
  Br & $2.3\ \%$ & $1.8\ \%$ & $9.9\ \%$\\ \hline
  Sr & $2.9\ \%$ & $2.5\ \%$ & $12.8\ \%$\\ \hline
  Zr & $1.4\ \%$ & $1.2\ \%$ & $6.1\ \%$\\ \hline
  \end{tabular}
  \caption{Die Abweichungen in Prozent.}
\end{table}
Die Abweichungen der Energien, Winkel und Abschirmkonstanten sind sehr gering, ausser bei Gallium. Das kann daran liegen, dass der Glanzwinkel unterhalb des unteren Plateaus und deshalb nicht auf der K-Kante liegt. Ausserdem ist der Verlauf der Kante nicht linear, weshalb eine lineare Regression hier eine sehr oberflächliche Annäherung ist. 

\subsection{Mosleysches Gesetz}
Die theoretische Rydbergkonstante ist $R_{\infty}=1.097\cdot 10^{7}\ \frac{1}{m}$. Im Vergleich mit der experimentellen ergibt sich eine Abweichung von 
\begin{equation*}
  R_{\%}=13\ \%
\end{equation*}
die dadurch erklärbar ist, dass die Konstante mit einer linearen Ausgleichsrechnung berechnet wurde. 

\section{Literaturverzeichnis}
[1] Technische Universität Dortmund, \textit{V602: Röntgenemission und -absorption}
[2] Bernhard Rupp, \textit{X-Ray Absorption Edge Energies} https://www.ruppweb.org/Xray/elements.html
