\section{Diskussion}
Bei den K-Linien weichen die Energien um
\begin{equation*}
  E_{\alpha_{\%}}=2\ \%
\end{equation*}
\begin{equation*}
  E_{\beta_{\%}}=1\ \%
\end{equation*}
und die Winkel um
\begin{equation*}
  \theta_{\alpha_{\%}}=2\ \%
\end{equation*}
\begin{equation*}
  \theta_{\beta_{\%}}=2\ \%
\end{equation*}
ab. Das ist wieder in einem geringen Rahmen und akzeptabel.

\subsection{Überprüfung der Bragg-Bedingung}
Wird der Theoriewert mit dem experimentellen Wert verglichen, ergibt sich zu
\begin{equation*}
  |\frac{\theta_{theo}-\theta_{exp}}{\theta_{exp}}|=0.08\%,
\end{equation*}
woraus sich erkennen lässt, dass die Abweichung sehr gering ist und die Bragg-Bedingung erfüllt.

\subsection{Emissionsspektrum}

\subsection{Absorptionsspektren}
Für die Energien, Winkel und Abschirmkonstanten ergeben sich folgende Abweichungen:
\begin{table}[H]
  \centering
  \begin{tabular}{l|l|l|l}
  & $E_{\%}$ & $\theta_{\%}$ & $\sigma_{\%}$\\ \hline
  Zn & $6\ \%$ & $5\ \%$ & $13\ \%$\\ \hline
  Ga & $7\ \%$ & $7\ \%$ & $7\ \%$\\ \hline
  Br & $1\ \%$ & $4\ \%$ & $12\ \%$\\ \hline
  Sr & $1\ \%$ & $1\ \%$ & $5\ \%$\\ \hline
  Zr & $7\ \%$ & $6\ \%$ & $14\ \%$\\ \hline
  \end{tabular}
  \caption{Die Abweichungen in Prozent.}
\end{table}
Die Abweichungen der Energien und Winkel sind sehr gering, die Abschirmungskonstanten jedoch überschreiten die $10\ \%$-Marke, was vermutlich durch die geringen Abweichungen in den Energien und Winkeln zustande kommt. 

\subsection{Mosleysches Gesetz}
Die theoretische Rydbergkonstante ist $R_{\infty}=1.097\cdot 10^{7}\ \frac{1}{m}$. Im Vergleich mit der experimentellen ergibt sich eine Abweichung von 
\begin{equation*}
  R_{\%}=13\ \%
\end{equation*}
die dadurch erklärbar ist, dass die Konstante mit einer linearen Ausgleichsrechnung berechnet wurde. 

\section{Literaturverzeichnis}
[1] Technische Universität Dortmund, \textit{V602: Röntgenemission und -absorption}
