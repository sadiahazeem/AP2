\section{Auswertung}
\label{sec:Auswertung}



\subsection{Überprüfung der Bragg-Bedingung}
\label{subsec:bragg}
Nach der Bragg-Bedingung ist das gemessene Intensitätsmaximum beim Glanzwinkel von ---- zu erwarten. \\
Dieser liegt bei dem verwendeten KBr-Kristall bei ---- und wird durch die in Tabelle aufgeführten -- Werte 
der Messung verifiziert ???. \\

------- tabelle --------



\subsection{Emissionsspektrum}
\label{subsec:emissionsspektrum}


\subsubsection*{Maximale Energie und minimale Wellenlänge}

Das charakteristische Spektrum der Kupfer-Röntgenröhre ist in Abbildung --- zu sehen.\\
Mit zunehmendem Winkel erkennt man den Grenzwinkel bei  $K_{\alpha}$ und $K_{\beta}$ 
Aus dem Grenzwinkel \begin{equation*}
  \theta_{min} = 
\end{equation*}
lassen sich die maximale Energie und die minimale Wellenlänge 
\begin{equation*}
  E_{max} = 
\end{equation*}
\begin{equation*}
  \lambda_{min} =
\end{equation*} 
berechnen.


\subsubsection*{Auflösungsvermögen der Apparatur}

Mit Hilfe der Halbwertsbreite lässt sich auch das Auflösungsvermögen der Apparatur bestimmen. \\
Die Halbwertsbreite berechnet sich aus den Winkeln $\theta_1 = $ und $\theta_2 = $.\\
So ergeben sich die Energien zu $E_1$ und $E_2$, aus deren Differenz $\Delta E = $ keV sich das Auflösungsvermögen
nach ----- zu $A = $ ergibt.


\subsubsection*{Abschirmkonstanten}

Aus den berechneten Energien $E_{K \alpha}$ und $E_{K \beta}$ und dem Literaturwert $E_{K,\;abs} = 8980.476 \; \mathrm{eV}$ können die 
Abschirmkonstanten $\sigma_1$, $\sigma_2$ und $\sigma_3$ von Kupfer wie folgt bestimmt werden. Die Ordnungszahl lautet $Z = 29$, $n=1$, $m=2$ und $l=3$. \\
Aus 
\begin{equation*}
  \sigma_1=Z-\sqrt{\frac{E_{Kabs}}{R_\infty}}
  \end{equation*}
  
  \begin{equation*}
  \sigma_2=Z-\sqrt{ \frac{m^2}{n^2}(Z-\sigma_1)^2 - \frac{m^2}{R_\infty} E_{K\alpha}}
  \end{equation*}
  
  \begin{equation*}
      \sigma_3=Z-\sqrt{ \frac{l^2}{n^2}(Z-\sigma_1)^2 - \frac{l^2}{R_\infty} E_{K\beta}}
  \end{equation*}
ergeben sie sich zu $\sigma_1 = 3,30°$, $\sigma_2 = 13,57°$ 
und $\sigma_3 = ??$. ------ gibt imaginäres ergebnis




\subsection{Absorptionsspektren}
\label{subsec:absorptionsspektrum}


\subsubsection*{Absorptionsspektrum von Zink}
In \autoref{fig:zink} ist das gemessene Absorptionsspektrum von Zink abgebildet.\\
Darin ist die K-Kante bei $\theta = 11,9°$ zu sehen. \\
Nach Gl. --- ist die ergibt sich die Absorptionsenergie $E_{Zn, \; K} = 9,138 \mathrm{ keV}$.\\
Daraus lässt sich die Abschirmkonstante $\sigma_{Zn, \; K} = 4,08°$ errechnen.
\begin{figure}
  \centering
  \includegraphics{zink.pdf}
  \caption{Absorptionsspektrum der Röntgenstrahlung von Zink.}
  \label{fig:zink}
\end{figure}


\subsubsection*{Absorptionsspektrum von Gallium}
Im Absorptionsspektrum von Gallium ist die K-Kante bei $\theta = 10,5°$ zu sehen. \\
Nach Gl. --- ist die ergibt sich die Absorptionsenergie $E_{Ga, \; K} = 10,34 \mathrm{ keV}$.\\
Daraus lässt sich die Abschirmkonstante $\sigma_{Ga, \; K} = 3,43°$ errechnen.

\subsubsection*{Absorptionsspektrum von Brom}
Im Absorptionsspektrum von Brom ist die K-Kante bei $\theta = 8,3°$ zu sehen. \\
Nach Gl. --- ist die ergibt sich die Absorptionsenergie $E_{Br, \; K} = 13,54 \mathrm{ keV}$.\\
Daraus lässt sich die Abschirmkonstante $\sigma_{Br, \; K} = 3,45°$ errechnen.

\subsubsection*{Absorptionsspektrum von Strontium}
Im Absorptionsspektrum von Strontium ist die K-Kante bei $\theta = 6,8°$ zu sehen. \\
Nach Gl. --- ist die ergibt sich die Absorptionsenergie $E_{Sr,\; K} = 15,9  \mathrm{ keV}$.\\
Daraus lässt sich die Abschirmkonstante $\sigma_{Sr, \; K} = 3,81°$ errechnen.

\subsubsection*{Absorptionsspektrum von Zirkonium}
Im Absorptionsspektrum von Zirkonium ist die K-Kante bei $\theta = 6,4°$ zu sehen. \\
Nach Gl. --- ist die ergibt sich die Absorptionsenergie $E_{Zr, \; K} = 16,9  \mathrm{ keV}$.\\
Daraus lässt sich die Abschirmkonstante $\sigma_{Zr, \; K} = 4,75°$ errechnen.





\subsubsection*{Moseleysches Gesetz}

Die lineare Ausgleichsrechnung ergibt für die Ausgleichsgerade $y = ax + b$ die Parameter $a = 3,443 \pm 0,2283$ keV und $b = -5,864 \pm 7,9946$ keV.\\
So ergibt sich für die experimentiell bestimmte Rydbergkonstante $R_{exp} = 9,56 \cdot 10^6 \mathrm{\frac{1}{m}}$. 

\begin{figure}
  \centering
  \includegraphics{plot_moseley.pdf}
  \caption{Die Quadratwurzel der Absorptionsenergie in Abhängigkeit von der Ordnungszahl Z mit Ausgleichsgeraden.}
  \label{fig:moseley}
\end{figure}






%\begin{figure}
%  \centering
%  \includegraphics{plot.pdf}
%  \caption{Plot.}
%  \label{fig:plot}
%\end{figure}

%\begin{figure}
%  \centering
%  \includegraphics{build/emissionsspektrum.pdf}
%  \caption{ noch einfügen }
%  \label{fig:plot2}
%\end{figure}



%\begin{figure}
%  \centering
%  \includegraphics{build/moseley.pdf}
%  \caption{Plot.}
%  \label{fig:plot3}
%\end{figure}
%
