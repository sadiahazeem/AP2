\section{Durchführung}
\label{sec:Durchführung}

\subsection{Aufbau}

Der Aufbau besteht aus einem Ultraschall Doppler-Generator im Pulsbetrieb,
2 MHz-Ultraschallsonden und einem Rechner für die Messung der Daten und deren Analyse mit FlowView. 
Es werden Strömungsrohre mit verschiedenen Innen- und Außendurchmessern verwendet.  
Die Strömungsflüssigkeit besteht aus einem Gemisch aus Wasser, Glycerin und Glaskugeln. 
Akustische Eigenschaften sind an die verwendete Ultraschallfrequenz angepasst. In diesem
Versuch werden laminare Strömungen benötigt, sodass die Viskosität der verwendeten Flüssigkeit
entsprechend gewählt wurde. Die Strömungsgeschwindigkeit lässt sich mithilfe einer Zentrifugalpumpe 
im Bereich von 0\;l/min bis 10\;l/min eingestellen.

\subsection{Messung}

Zunächst wird die Strömungsgeschwindigkeit untersucht. Die Kopplung der Sonden an das Strömungsrohr 
erfolgt mit den Dopplerprismen und Ultraschall-Gel.\\
\noindent Es wird die Maximalgeschwindigkeit an der Zentrifugalpumpe eingestellt und notiert, sodass später eingestellte 
Strömungsgeschwindigkeiten in Prozent von der Maximalgeschwindigkeit bzw. Pumpleistung angegeben werden können. Nun 
wird für verschiedene Strömungsgeschwindigkeiten die Frequenzverschiebung $\Delta\nu$ notiert.
Dies wird jeweils für alle drei Dopplerwinkel mit fünf Strömungsgeschwindigkeiten durchgeführt.\\
\newline 
\noindent Im zweiten Teil der Messung soll das Strömungsprofil ermittelt werden. Es werden bei 70\;\% Pumpleistung mit 
3/8 Zoll Schlauch und 15° Dopplerwinkel die Strömungsgeschwindigkeit und Streuintensitätswert gemessen. Die Messung findet
ab 30\;mm bis 11\;mm Messtiefe in 0,75\;mm Schritten statt.