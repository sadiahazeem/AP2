\section{Zielsetzung}
Ziel des Versuches ist die Untersuchung der Temperaturabhängigkeit der thermischen Elektronenemission und die 
Bestimmung der materialspezifischen Austrittsarbeit von Wolfram. 

\section{Theorie}
\label{sec:Theorie}


\subsection{Richardson-Gleichung}
}\begin{equation}
  j_{\symup{S}}(T) = 4 \pi \frac{e_0 m_0 k^2}{h^3} T^2 \exp\left(\frac{-e_0 \phi}{k T} \right),
  \label{eq:richardson}
\end{equation}
mit der aus der Temperatur $T$ sowie den Konstanten $e_0$ (Elementarladung),
$m_0$ (Elektronenruhemasse), $k$ (Boltzmann-Konstante) und
$h$ (Planksches Wirkungsquantum) die Sättigungsstromdichte, also den Strom der Elektronen
aus der Metalloberfläche, berechnet werden.\\
\cite{sample}
