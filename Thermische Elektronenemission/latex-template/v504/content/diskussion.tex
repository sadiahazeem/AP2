\section{Diskussion}
\label{sec:Diskussion}
Der Vergleich des theoretischen und experimentellen Exponenten des Langmuir-Schottkyschen-Gesetzes ergibt eine Abweichung von
\begin{equation*}
  |\frac{1.5-1.325}{1.325}|=13.2\ \%,
\end{equation*}
die relativ gering ist und dadurch erklärt werden kann, dass für die Berechnung des Wertes, die Funktion auf eine Exponentialfunktion gefittet wurde, die in Relation zum Strom und nicht zur Stromdichte steht.\\
Das Ausrechnen der Kathodentemperatur ist durch den direkten Weg in \autoref{sec:T2} theoretisch genauer als durch den Fit in \autoref{sec:t1}. Allerdings wäre da eine direkte Temperaturmessung an der Kathode notwendig, um das verifizieren zu können.\\
Der Literaturwert der Austrittsarbeit von Wolfram liegt bei $W_{A}=4.54$ eV. Der Vergleich mit dem experimentellen Wert ergibt eine Abweichung von
\begin{equation*}
  |\frac{4.54-3.616}{3.616}|=25.6\ \%,
\end{equation*}
die sich durch Ableseungenauigkeit und der Anfälligkeit der Messaperatur durch Stöße erklären lässt. Zusätzlich lässt sich beim Anlaufstrom das menschliche Magnetfeld als Fehlerquelle ermitteln, da das Nanoamperemeter auf kleinste Veränderung dessen reagiert.
