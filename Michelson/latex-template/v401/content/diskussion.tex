\section{Diskussion}
\label{sec:Diskussion}

Der experimentiell bestimmte Wert $\lambda = \SI{644,76}{\nano\metre}$ für die Wellenlänge des Lasers weicht um 
\begin{equation*}
    \eta_\lambda = 1,57\%
\end{equation*}
vom auf dem Gerät angegebenen Wert $\lambda = 635$ nm ab. \\
Dies bestätigt, dass Messungen mit dem Michelson-Interferometer sehr präzise sind.\\
Das zeigt sich auch durch den experimentiell bestimmten Brechungsindex $n = 1,0000000280$ von Luft, der um 
\begin{equation*}
    \eta_n = 0,03\%
\end{equation*}
vom Literaturwert $n = 1,000292$ abweicht.\\

\section{Inhaltsverzeichnis}

[1] Technische Universität Dortmund, \textit{V401: Das Michelson-Interferometer}

\section{Anhang}

\begin{figure}[H]
  \centering
  \includegraphics[width=7cm]{anhang}
  \caption{Originale Messdaten}
  \label{fig:anh}
\end{figure}
