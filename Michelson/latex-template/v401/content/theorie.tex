\section{Theorie}
\label{sec:Theorie}
\section{Ziel}

In diesem versuch soll die Länge und die Wellenlänge eines Lasers gemessen sowie der Brechungsindex von Luft mit dem Michelson-Interferometer ermittelt werden.

\section{Theorie}

\subsection{Inteferenz und Kohärenz}
Licht kann als elektromagnetische Welle angenommen werden. Diese lässt sich mit der elektrischen Feldstärke 
\begin{equation}
  \vec E(x,t)=\vec E_{0}cos(kx-\omega t-\delta) \nonumber
\end{equation}
beschreiben, bei der x der Ort, t die Zeit, k die Wellenzahl, $\omega$ die Kreisfrequenz und $\delta$ die Phasenabweichung am Nullpunkt ist. Die Lichtwellenausbreitung lässt sich dann mit den Maxwellschen-Gleichungen beschreiben. Für jedes E-Feld gilt das Superpositionsprinzip. Da jedoch das E-Feld bei hohen Frequenzen schwer messbar ist, wird die Intensität 
\begin{equation}
  I=const|\vec E|^2 \nonumber
\end{equation}
genutzt.Insofern zwei E-Felder superponieren, kann die Intensität $I_{Ges}$ durch
\begin{equation}
  I_{Ges}=\frac{const}{t_{2}-t_{1}} \int^{t_{2}}_{t_{1}} |\vec E_{1}+\vec E_{2}|^2 (x,t)dt \nonumber
\end{equation}
berechnet werden. Damit ergibt sich 
\begin{equation}
  I_{Ges}=2const\vec E_{0}^2 (1+cos(\delta_{2}-\delta_{1})) \nonumber
\end{equation}
für den Ansatz $\vec E_{1/2}=\vec E_{0}e^{i(kx-\omega t -\delta_{1/2})}$. Hierbei ist der zweite Summand
\begin{equation}
  2const\vec E_{0}^2 cos(\delta_{2}-\delta_{1}) \nonumber
\end{equation}
der \textit{Interferenzterm}.
Die Interferenz kann konstruktiv oder destruktiv verlaufen. Sie verschwindet insofern 
\begin{equation}
  \delta_2 - \delta_1 = (2n+1)\pi \quad n \in \mathrm{N_{0}} \nonumber
\end{equation}
ist. Allerdings ist Licht, dass aus zwei Lichtquellen stammt, nicht interferenzfähig und wird als \textit{inkohärent} bezeichnet. Deswegen muss \textit{kohärentes} Licht erzeugt werden, das mit Lasern realisierbar ist.
Wenn man das Licht mit einem Strahlteiler teilt und einen Strahl durch einen Spiegel reflektiert, sodass beide an der selben Stelle des Schirms auftreffen, wie in \autoref{fig:1}, dann sorgt die durch die Weglänge unterschiedliche Phasendifferenz für ein Interferenzmuster.
\begin{figure}[H]
  \centering
  \includegraphics[width=7cm]{1}
  \caption{Mögliche Versuchsanordnung [V401]}
  \label{fig:1}
\end{figure}
Der maximale Weglängenunterschied, bei denen die Lichtstrahlen noch ein stabiles Interferenzmuster abbilden, wird \textit{Koheränzlänge} $\ell$ genannt mit
\begin{equation}
  \ell=N\lambda, \nonumber
\end{equation}
wobei N die maximal beobachteten Intensitätsmaxima auf dem Schirm sind und $\lambda$ die Wellenlänge des Lasers.

\subsection{Michelson-Interferometer}

Mit dem Michelson-Interferometer können optische Größen gemessen werden. 
\begin{figure}[H]
  \centering
  \includegraphics[width=7cm]{2}
  \caption{Aufbau des Michelson-Interferometers [V401]}
  \label{fig:2}
\end{figure}
Das Licht strahlt auf eine halbdurchlässige Platte, die einen Teil des Lichstrahls durchlässt und die andere orthogonal spiegelt. Beide Teilstrahlen treffen auf die Spiegel 1 und 2 aus \autoref{fig:2} die die Strahlen zurückreflektieren, die dann auf den Detektor fallen. Beide Strahlenbündel sind kohärent, weil $\bar{S_{1}P}=\bar{S_{2}}<\bar{LP}$ gesetzt wird. Bei $\bar{S_{1}P}=\bar{S_{2}}$ ist der Gangunterschied am Detektor $\frac{\lambda}{2}$ und die Lichtbündel inteferieren destruktiv. Wird ein Spiegel jedoch um den Weg $\Delta d$ verschoben, dann ändert sich die Intensität des Interferenzmusters am Detektor und es ergibt sich die Beziehung
\begin{equation}
  \Delta d=N\frac{\lambda}{2}
  \label{eq:1}
\end{equation}
zwischen der Verscheibung und der Wellenlänge [1].\\
Eine andere Methode den Wegunterschied zu ändern ist, das Licht durch ein Medium mit verändertem Brechungsindex zu senden. Wenn das Medium die Länge $b$ hat und der Brechungsindex sich zu $n+\Delta n$ verschiebt, ist der Wegunterschied der beiden Strahlen $b\Delta n$. Die Änderung des Brechungsindex kann durch Evakuierung einer Messbox oder Erhöhung des Drucks realisiert werden, in der sich das zu untersuchende Medium befindet. Die Beziehung \eqref{eq:1} ändert sich zu
\begin{equation}
  n\Delta n =N\frac{\lambda}{2} \nonumber
\end{equation}
und lässt sich zu 
\begin{equation}
  n=\sqrt{1+f(\lambda)N_{T}}
  \label{eq:2}
\end{equation}
umformen, mit $N_{T}$ als Anzahl von Molekülen, die durch die Lichwellen der Wellenlänge $\lambda$ angeregt wurden. Im optisch sichtbaren Bereich lässt sich \eqref{eq:2} in
\begin{equation}
  n=1+\frac{f}{2}N-... \nonumber
\end{equation}
entwickeln. Für Gase zwischen 0 und 1 Bar gilt die ideale Gasgleichung
\begin{equation}
  pV=RT 
  \label{eq:3}
\end{equation}
Um die Brechungsindexänderung $\Delta n(p,p')$ zu berechnen, wird 
\begin{equation}
  \Delta n(p,p')=f/2(N(p,T)-N(p',T))
  \label{eq:4}
\end{equation}
verwendet, bei der 
\begin{equation}
  N(p,T)=\frac{p}{T}\frac{T_{0}}{p_{0}}N_{L}, \quad N_{L}=2,687\cdot 10^25\frac{1}{m^3} \nonumber
\end{equation}
\begin{equation}
  N(p',T)=\frac{p'}{T}\frac{T_{0}}{p_{0}}N_{L}, \quad N_{L}=2,687\cdot 10^25\frac{1}{m^3}
  \label{eq:5}
\end{equation}
die Anzahl der Moleküle in Abhängigkeit des Drucks p (oder des geänderten Drucks p') und der Temperatur T ist.
Mit \eqref{eq:5} folgt aus \eqref{eq:4}
\begin{equation}
  \Delta n(p,p')=\frac{f}{2}\frac{N_{L}}{T}\frac{T_{0}}{p_{0}}(p-p') 
  \label{eq:5}
\end{equation}
und damit
\begin{equation}
  n(p_{0},T_{0})=1+\Delta n(p,p')\frac{T}{T_{0}}\frac{p_0}{p-p'},
  \label{eq:6}
\end{equation}
der Brechungsindex des Gases bei Normaldruck und -Temperatur [1].
