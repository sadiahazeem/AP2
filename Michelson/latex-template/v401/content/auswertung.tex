\section{Auswertung}
\label{sec:Auswertung}

\subsection{Bestimmung der Wellenlänge}

Die Wellenlänge des Lasers wird mit Gleichung \autoref{{eq:deltad}} bestimmt. 
Dabei wird der Wert für die Verschiebung wie folgt um die Hebelübersetzung $C_{Hü} = 5,046$ korrigiert 
\begin{equation*}
  \Delta d_ü = \frac{\Delta d}{C_{Hü}} .
\end{equation*}

Die Messwerte sowie die daraus bestimmten Wellenlängen sind in \autoref{tab:wellenl} aufgelistet.\\

\begin{table}
    \centering
    \caption{Die aufgenommenen Messwerte und daraus bestimmten Wellenlängen.}
    \label{tab:wellenl}
    \begin{tabular}{c c c}
      \toprule
      {$\increment d$ mm} & {$N$} & {$\lambda$ nm} \\
      \midrule
      5,0     &       3100 $\pm$ 5    &   639,280 $\pm$ 1,031\\
      5,0     &       3120 $\pm$ 5    &   635,182 $\pm$ 1,018\\
      5,0     &       3100 $\pm$ 5    &   639,280 $\pm$ 1,031\\
      5,0     &       3070 $\pm$ 5    &   645,527 $\pm$ 1,051\\
      5,0     &       3090 $\pm$ 5    &   641,349 $\pm$ 1,038\\
      5,0     &       3080 $\pm$ 5    &   643,431 $\pm$ 1,045\\
      5,0     &       3010 $\pm$ 5    &   658,395 $\pm$ 1,093\\
      5,0     &       3090 $\pm$ 5    &   641,349 $\pm$ 1,038\\
      5,0     &       3030 $\pm$ 5    &   654,049 $\pm$ 1,079\\
      5,0     &       3050 $\pm$ 5    &   649,760 $\pm$ 1,065\\
      \bottomrule
    \end{tabular}
  \end{table}
  \noindent

  Durch die Mittelwertberechnung mithilfe von python \cite{uncertainties} ergibt sich für die Wellenlänge des Lasers
\begin{equation*}
  \lambda = \SI{644,76 \pm 0,33}{\nano\metre}.
\end{equation*}



\subsection{Bestimmung des Brechungsindices von Luft}

Zur Berechnung wird die \autoref{eq:brechin} genutzt. Die Länge der Messkammer beträgt $b = 50 \cdot 10^-3$ m.\\
Die Druckdifferenz, die gemessene Impulsanzahl $N$ sowie der daraus bestimmte Brechungsindex sind in \autoref{tab:Brechungsindex} aufgelistet.\\
Es wurde abwechselnd der Druck in der Messkammer um die angegebene Druckdifferenz $\increment p$ erniedrigt und erhöht.\\

  \begin{table}[h]
    \centering
    \caption{Die aufgenommenen Messwerte und daraus bestimmten Brechungsindices.}
    \label{tab:Brechungsindex}
    \begin{tabular}{c c c}
      \toprule
      {$\increment p$ [$\si{\bar}$]} & {$N$} & {Brechungsindex $n$} \\
      \midrule
      0,8    &   48  &  1,000 \\
      0,8    &   47  &  1,000 \\
      0,8    &   60  &  1,000 \\
      0,8    &   45  &  1,000 \\
      0,8    &   44  &  1,000 \\
      0,8    &   41  &  1,000 \\
      
      \bottomrule
    \end{tabular}
  \end{table}
  \noindent
  Durch die Berechnung des Mittelwerts ergibt sich für den Brechungsindex von Luft
  \begin{equation*}
    n = 1,0000000280 \pm 0,0000000012.
  \end{equation*}