\section{Diskussion}
\label{sec:Diskussion}

% abweichung der mit R bestimmten reichweiten von den durch energieverlust 
% bestimmten reichweiten

Die mittlere Reichweite $x_{\mathrm{30mm}} = 23,56 \mathrm{mm}$ weicht um 
$\eta_{\mathrm{x}} = 0,84 \%$ von $x_{\mathrm{36mm}} = 23,76 \mathrm{mm}$ ab.\\

Die Energien
\begin{eqnarray}
    E_{\mathrm{R, \; 30mm}} &=& 20,96 \; \mathrm{MeV}  \nonumber \\
    E_{\mathrm{R, \; 36mm}} &=& 21,22 \; \mathrm{MeV}  \nonumber
\end{eqnarray}
weichen um $\eta_{\mathrm{E}} = 1,22 \%$ voneinander ab.\\
Diese Abweichung ist als eher gering einzustufen.
Mögliche Fehlerquellen liegen unter Anderem in der Aufnahme der Messwerte, 
da beispielsweise das Ablesen des Abstands zwischen Probe und Halbleiter-Sperrschichtzähler sowie des Drucks sehr unpräzise ist.\\
Bei der Auswertung der Statistik des radioaktiven ist zu beobachten, dass die 
Gaußglocke deutlich passender über der Verteilung der Messwerte liegt. 
Der Peak der gemessenen Verteilung liegt niedriger und ist flacher, als der der Poissonverteilung.
Außerdem sind die gemessenen Werte breiter gestreut, als die poissonverteilten.\\