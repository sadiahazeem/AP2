\section{Auswertung}
\label{sec:Auswertung}

\subsection{Reflexionsgesetz}
Die Messungen werden in \autoref{tab:1} zusammengefasst.
\begin{table}[H]
  \centering
  \caption{Die gemessenen Einfalls- und Ausfallswinkel.}
  \begin{tabular}{l|l}
  $\alpha_1$/° & $\alpha_2$/°\\\hline
  20 & 20\\
  25 & 25\\
  30 & 30\\
  35 & 35\\
  40 & 40\\
  45 & 45\\
  50 & 50\\\hline
  \end{tabular}
  \label{tab:1}
\end{table}

\subsection{Brechungsgesetz}
Das Brechungsgesetz wurde an einer aus Plexiglas bestehenden planplanaren Platte überprüft. Es wurden die Werte aus \autoref{tab:2} gemessen und mit \eqref{eq:sn} der zugehörige Brechungsindex sowie die Ausbreitungsgeschwindigkeit des Lichtes im Plexiglas berechnet.
\begin{table}[H]
  \centering
  \caption{Einfallswinkel des Lasers, Brechungswinkel im Plexiglas, Brechungsindizes des Plexiglases und Ausbreitungsgeschwindigkeit des Lichtes im Plexiglas.}
  \begin{tabular}{l|l|l|l}
  $\alpha$/° & $\beta$/° & n & $v\cdot 10^8$/(s/m)\\\hline
  10 & 7.5 & 1.33 & 2.2534\\
  20 & 13.5 & 1.47 & 2.0462\\
  25 & 17 & 1.45 & 2.0740\\
  30 & 21 & 1.40 & 2.1487\\
  35 & 23 & 1.47 & 2.0422\\
  40 & 26 & 1.47 & 2.0445\\
  45 & 28 & 1.51 & 1.9904\\\hline
  \end{tabular}
  \label{tab:2}
\end{table}
Werden die Brechungsindizes in 
\begin{align}
  \bar n&=\frac{1}{m}\sum\limits_{i=1}^m n_i \nonumber\\
  \Delta \bar n&=\sqrt{\frac{1}{m(m-1)}\sum\limits_{i=1}^m (n_i-\bar n)^2}
  \label{eq:mit}
\end{align}
eingesetzt, ergibt sich der Mittelwert und Fehler
\begin{equation*}
  \bar n \pm \Delta \bar n=1.44 \pm 0.02
\end{equation*}
für den Brechungsindex von Plexiglas.\\
Mit den selben Formeln aus \eqref{eq:mit} ergibt sich der Mittelwert und Fehler
\begin{equation*}
  \bar v \pm \Delta \bar v=2.0856 \pm 0.03
\end{equation*}
der Ausbreitungsgeschwindigkeit des Lichtes im Plexiglas.

\subsection{Prisma}
Während des Experiments wurden die Einfallswinkel $\alpha_1$ und Ausfallswinkel $\alpha_2$ aus \autoref{tab:3} für den grünen und roten Laser gemessen.
\begin{table}[H]
  \centering
  \caption{Einfallswinkel und Ausfallswinkel für den grünen und roten Laser.}
  \begin{tabular}{l|l|l|l}
  $\alpha_{1_g}$/° & $\alpha_{2_g}$/° & $\alpha_{1_r}$/° & $\alpha_{2_r}$/°\\\hline
  30 & 67 & 25 & 73\\
  35 & 59 & 30 & 63\\
  40 & 53 & 35 & 59\\
  45 & 47 & 40 & 51\\
  50 & 41 & 45 & 46\\\hline
  \end{tabular}
  \label{tab:3}
\end{table}
Aus \eqref{eq:sn} und $\alpha_1$ lassen sich die jeweiligen Brechungswinkel für $\beta_1$ und mit der Winkelbeziehung $\gamma=\beta_1 +\beta_2$ mit $\gamma=60^\circ$ die Brechungswinkel $\beta_2$ berechnen. Diese werden in \autoref{tab:4} zusammengefasst.
\begin{table}[H]
  \centering
  \caption{Brechungswinkel für den grünen und roten Laser.}
  \begin{tabular}{l|l|l|l}
  $\beta_{1_g}$/° & $\beta_{2_g}$/° & $\beta_{1_r}$/° & $\beta_{2_r}$/°\\\hline
  19.61 & 40.39 & 16.48 & 43.52\\
  22.64 & 37.36 & 19.61 & 40.39\\
  25.56 & 34.44 & 22.64 & 37.36\\
  28.33 & 31.67 & 25.56 & 34.44\\
  30.94 & 29.06 & 28.33 & 31.67\\\hline
  \end{tabular}
  \label{tab:3}
\end{table}
Mit den Informationen aus \autoref{tab:2} und \autoref{tab:3} errechnen sich die Ablenkungen aus \autoref{tab:4}.
\begin{table}[H]
  \centering
  \caption{Ablenkungen für die Werte aus \autoref{tab:2} und \autoref{tab:3}.}
  \begin{tabular}{l|l|l}
  & $\delta_{g}$ & $\delta_{r}$\\\hline
  $\delta_{1}$ & 7 & 8\\
  $\delta_{2}$ & 4 & 3\\
  $\delta_{3}$ & 3 & 4\\
  $\delta_{4}$ & 2 & 1\\
  $\delta_{5}$ & 1 & 1\\\hline
  \end{tabular}
  \label{tab:4}
\end{table}
Aus \autoref{tab:4} ergeben sich mit \eqref{eq:mit} die Mittelwerte und Fehler für die Ablenkungen
\begin{align*}
  \bar \delta_g \pm \Delta \bar \delta_g&=3.40 \pm 1.03\ \textrm{und}\\
  \bar \delta_r \pm \Delta \bar \delta_r&=3.40 \pm 1.29
\end{align*}
des grünen und roten Lasers.

\subsection{Beugung am Gitter}
Die Wellenlänge der Laser kann durch 
\begin{equation*}
  \lambda=d\frac{\sin(\varphi)}{k}
\end{equation*}
berechnet werden, wobei $d$ die Gitterkonstante, $k$ die Beugungsordnung und $\varphi$ der Beugungswinkel ist. In \autoref{tab:5} werden die Beugungsordnungen, Winkel und Wellenlängen für die jeweiligen Gitter und Laser zusammengefasst.
\begin{table}[H]
  \centering
  \caption{Beugungsordnungen, Winkel und Wellenlängen der jeweiligen Gitter und Laser}
  \begin{tabular}{l|l|l|l|l|l}
  d/$\mu$m & $k$ & $\varphi_g$ /°& $\varphi_r$/° & $\lambda_g$ & $\lambda_r$\\\hline
  10 & 1 & 1 & 0.5 & 174.50 nm & 87.30 nm \\
     & 2 & 4 & 14 & 348.80 nm & 1.21 $\mu$m  \\
     & 3 & 7 & 27.5 & 406.20 nm & 1.54 $\mu$m \\\hline
  3.33 & 1 & 2.5 & 9.5 & 145 nm & 550 nm\\
     & 2 &17 & 20.5 & 486.80 nm & 583 nm \\\hline
  1.67 & 1 & 19.5 & 22 & 557 nm & 625.60 nm\\\hline
  \end{tabular}
  \label{tab:5}
\end{table}
Mit den Wellenlängen aus \autoref{tab:5} können die Mittelwerte und Fehler durch \eqref{eq:mit} zu
\begin{align*}
  \bar \lambda_{g_10} \pm \Delta \bar \lambda_{g_10}&=(309.83 \pm 69.83)\ \textrm{nm},\\
  \bar \lambda_{g_3.33} \pm \Delta \bar \lambda_{g_3.33}&=(315.9 \pm 170.9)\ \textrm{nm},\\
  \bar \lambda_{g_1.67} \pm \Delta \bar \lambda_{g_1.67}&=557\ \textrm{nm},\\
  \bar \lambda_{r_10} \pm \Delta \bar \lambda_{r_10}&=(945.8 \pm 439.7)\ \textrm{nm},\\
  \lambda_{r_3.33} \pm \Delta \bar \lambda_{r_3.33}&=(566.5 \pm 16.5) \textrm{nm}\ \textrm{und}\\
  \lambda_{r_1.67} \pm \Delta \bar \lambda_{r_1.67}&=625.6\ \textrm{nm}
\end{align*}
für den grünen und roten Laser berechnet werden. Insgesamt ergibt sich mit \eqref{eq:mit} für die Wellenlänge der Mittelwert und Fehler 
\begin{align*}
  \bar \lambda_g \pm \Delta \bar \lambda_g&=(394.18 \pm 81.43)\ \textrm{nm}\ \textrm{und}\\
  \bar \lambda_r \pm \Delta \bar \lambda_r&=(712.63 \pm 117.83)\ \textrm{nm}
\end{align*}
für den grünen und roten Laser.
