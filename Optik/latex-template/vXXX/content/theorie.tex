\section{Zielsetzung}
Ziel des Versuches ist die Untersuchung grundlegender Phänomene der Optik, wie zum Beispiel 
das Snellius'sche Gesetz, Relflexion, Beugung und Brechung.\\



\section{Theorie}
\label{sec:Theorie}

\subsection*{Reflexion}
Das Reflexionsgesetz besagt, dass ein Strahlenbündel, das auf eine Grenzfläche trifft, im Winkel
\begin{equation*}
    \alpha_1 = \alpha_2
\end{equation*}
reflektiert wird.\\
Hier ist $\alpha_1$ der Einfallswinkel, $\alpha_2$ der Ausfallswinkel.\\

\subsection*{Brechung}
Die Brechung kann durch das Gesetz von Snellius beschrieben werden. Das besagt, dass
\begin{equation*}
    n_1 \cdot sin(\alpha) = n_2 \cdot sin(\beta)
\end{equation*}
gilt. Hierbei ist $n_1$ der Brechungsindex des ersten Mediums, $\alpha$ der Einfallswinkel zum Lot im ersten Medium.\\
$n_2$ und $\beta$ die entsprechenden Größen im zweiten Medium.\\


\subsection*{Reflexion und Transmission}
Häufig wird das Strahlenbündel weder vollständig reflektiert noch transmittiert.\\
Es wird, abhängig vom Material 