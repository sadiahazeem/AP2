\section{Diskussion}
\label{sec:Diskussion}

\subsection{Reflexionsgesetz}
Das Reflexionsgesetz wird eindeutig durch das Experiment bestätigt.

\subsection{Brechungsgesetz}
Der Brechungsindex von Plexiglas ist 
\begin{equation*}
  n=1.49
\end{equation*}
und wird dieser mit dem aus dem Experiment ermittelten Brechungsindex verglichen, ergibt sich die Abweichung
\begin{equation*}
  |\frac{n_{th} - n_{exp}}{n_{exp}}|=0.035=3.5\%,
\end{equation*}
die noch im Rahmen liegt \cite{2}. Diese Abweichung kann durch die Papierschablone verursacht worden sein, bei der sehr leicht Abgleichungs- oder Ablesefehler entstehen können.\\
Die theoretische Geschwindigkeit im Plexiglas
\begin{equation*}
  v_2=2.0120\cdot 10^8 \frac{\textrm{m}}{\textrm{s}}
\end{equation*} 
ergibt sich aus \autoref{eq:teil2}. Im Vergleich mit der experimentell bestimmten Geschwindigkeit des Lichtes im Plexiglas, ergibt sich 
\begin{equation*}
  |\frac{v_{th} - v_{exp}}{v_{exp}}|=0.035=3.5\%
\end{equation*}
für die Abweichung der Geschwindigkeit. Auch hier musste eine Papierschablone benutzt werden, die die gleichen Mali wie oben aufweist.

\subsection{Ablenkung im Prisma}
Zur Ablenkung kann keine Aussage getroffen werden, weil es keinen Vergleichswert gibt. 

\subsection{Beugung am Gitter}
Nach \cite{1} ist die Wellenlänge 
\begin{align*}
  \lambda_g&=532\ \textrm{nm}\ \textrm{und}\\
  \lambda_r&=635\ \textrm{nm}
\end{align*}
für den grünen und roten Laser. Im Vergleich mit den experimentellen Werten fällt sofort eine große Abweichung
\begin{align*}
  |\frac{\lambda_{g_{th}} - \lambda_{g_{exp}}}{\lambda_{g_{exp}}}|&=0.350=35.0\ \% \\
  |\frac{\lambda_{r_{th}} - \lambda_{r_{exp}}}{\lambda_{r_{exp}}}|&=0.109=10.9\ \%
\end{align*}
für die Wellenlängen der Laser auf. Das kann durch die mehreren Papierschablonen verursacht worden sein. Zusätzlich können auch die Laser oder die Gitter falsch platziert worden sein.
