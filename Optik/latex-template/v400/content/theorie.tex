\section{Zielsetzung}
Ziel des Versuches ist die Untersuchung grundlegender Phänomene der Optik, wie zum Beispiel 
das Snellius'sche Gesetz, Relflexion, Beugung und Brechung.\\



\section{Theorie}
\label{sec:Theorie}
\subsection{Geometrische Optik}

\subsubsection*{Reflexion}
Das Reflexionsgesetz besagt, dass ein Strahlenbündel, das auf eine Grenzfläche trifft, im Winkel
\begin{equation*}
    \alpha_1 = \alpha_2
    \label{eq:teil1}
\end{equation*}
reflektiert wird.\\
Hier ist $\alpha_1$ der Einfallswinkel, $\alpha_2$ der Ausfallswinkel.\\

\subsubsection*{Brechung}
Die Brechung kann durch das Gesetz von Snellius beschrieben werden. Das besagt, dass
\begin{equation}
    n_1 \cdot sin(\alpha) = n_2 \cdot sin(\beta)
    \label{eq:teil2}
\end{equation}
gilt. Hierbei ist $n_1$ der Brechungsindex des ersten Mediums, $\alpha$ der Einfallswinkel zum Lot im ersten Medium.\\
$n_2$ und $\beta$ die entsprechenden Größen im zweiten Medium.\\


\subsubsection*{Reflexion und Transmission}
Häufig wird das Strahlenbündel weder vollständig reflektiert noch transmittiert.\\
Es wird, abhängig vom Material, ein Teil $T$ der Intensität transmittiert, Teil $R$ reflektiert. 
Die beiden Anteile stehen in der Relation \begin{equation*}
    T + R = 1
\end{equation*}
zueinander, da die gesamte Intensität im System erhalten bleibt.\\

\subsection{Wellenoptik}
\subsubsection*{Licht als elektromagnetische Welle}
Eine elektromagnetische Welle wird durch ihre Frequenz $\nu$, Ausbreitungsgeschwindigkeit $v$ und 
Wellenlänge $\lambda$ charakterisiert. \\
Für die Intensitäten gilt das Superpositionsprinzip, sodass die Intensitäten konstruktiv oder destruktiv interferierender Wellen addiert bzw. subtrahiert werden können.\\
Die Art der Interferenz hängt vom Phasenversatz ab.\\

\subsubsection*{Beugung am Gitter}
Die Beugung am Gitter veranschaulicht das Phänomen der Interferenz.\\ 
Ihr liegt das Huygenssche Prinzip zugrunde, das besagt „Jeder Punkt einer Welle ist der Ausgangspunkt einer Elementarwelle gleicher Frequenz. 
Die Einhüllende aller Sekundärwellen stellt zu einem späteren Zeitpunkt die neue Lage der Wellenfront dar." \cite{sample}\\
Trifft das am Gitter gebeugte Licht auf einen Schirm, werden Intensitätsmaxima in periodeischen Abständen sichtbar. 
Sie können am Einzelspalt der Breite $b$ mit der Formel
\begin{equation*}
    b \cdot sin(\alpha) = k \cdot \lambda
\end{equation*}
beschrieben werden. Hier ist $\lambda$ die Wellenlänge, k der Index des 
Maximums und $\alpha$ der Winkel relativ zur geradlinigen Ausbreitungsrichtung.\\
Für Gitter mit mehr als einem Spalt wird die Formel zu 
\begin{equation*}
    d \cdot sin(\alpha) = k \cdot \lambda.
\end{equation*}
Dabei ist d die Gitterkonstante.\\

