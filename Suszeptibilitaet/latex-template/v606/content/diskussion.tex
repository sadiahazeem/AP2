\section{Diskussion}
\label{sec:Diskussion}

Es wurde mit einer Güte von $Q=20$ gemessen. Die ausgerechnete Güte beträgt allerdings $Q=3.975$. Es lässt sich leicht erkennen, dass das eine große Abweichung ist. 
Beim Vergleich der Suszeptibilitäten 
\begin{align*}
  |\frac{\chi_{th_{N}}-\chi_{exp_{N}}}{\chi_{exp_{N}}}|&=27\ \%\\
  |\frac{\chi_{th_{D}}-\chi_{exp_{D}}}{\chi_{exp_{D}}}|&=15\ \%\\
  |\frac{\chi_{th_{G}}-\chi_{exp_{G}}}{\chi_{exp_{G}}}|&=3\ \%
\end{align*}
wird bemerkbar, dass die ersten beiden Stoffe zu extrem abweichen und die Auswertung keine reliable Zustimmung der Theorie ist und dementsprechend beim Messvorgang etwas falsch abgelesen wurde, zum Beispiel die 10 mV Skala mit der 30 mV Skala verwechselt wurde, oder die Rundungen für die starken Abweichungen gesorgt haben. Beim dritten Molekül ergibt sich alles wunderbar und die Abweichung ist noch im Signifikanzintervall, bestätigt also als Einziges die Theorie. 
