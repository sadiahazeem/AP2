\section{Ziel}
\label{sec:Ziel}

Ziel des Versuches ist die Bestimmung der Suszeptibilitäten
verschiedener paramagnetischer Oxide seltener Erd-Elemente.\\
Dazu wird eine Brückenschaltung verwendet.\\ 

%\cite{sample}

\section{Theorie}
\label{sec:Theorie}

\subsection{Magnetische Suszeptibilität}

Die magnetische Suszeptibilität $\chi$ gibt Auskunft darüber, wie sehr sich 
die Magnetisierung $\vec{M}$ des Stoffes in einem äußeren Magnetfeld ändert.\\
$\chi$ ist dimensionslos und hängt von verschiedenen Parametern, wie der magnetischen Feldstärke
$\vec{H} = \frac{1}{\mu_0} \vec{B}$ und der Temperatur $T$ ab. \\
\begin{equation}
    \vec{M} = \mu_0 \chi \vec{H}
\end{equation}
Bei Raumtemperatur und in einem kleinen Magnetfeld $\vec{B} < 1 \mathrm{T}$ sind die Suszeptibilitäten
näherungsweise konstant. \\
Stoffe, deren Suszeptibilität unter Null liegt, sind diamagnetisch. Unter Einwirkung eines äußeren Magnetfeldes
magnetisiert sich ihr inneres entgegengesetzt zum äußeren Feld, sodass 
das Magnetfeld im Inneren des Volumens schwächer ist als außen. \\
Wenn die Suszeptibilität eines Stoffes über Null liegt, ist er paramagnetisch. 
Das innere Magnetfeld ist bei einem extern anliegenden Magnetfeld stärker als das externe, anregende Magnetfeld.\\



\subsection{Suszeptibilität paramagnetischer Stoffe}

Paramagnetische Stoffe haben einen nicht verschwindenen Drehimpuls.\\
Der Gesamtdrehimpuls $\vec{J}$ setzt sich aus dem Gesamtspin $\vec{S}$ und der Summe $\vec{L}$ der Bahndrehimpulse zusammen.\\
Ist die Elektronenschale mehr als halbvoll besetzt, gilt $\vec{J} = \vec{L} - \vec{S}$, sonst $\vec{J} = \vec{L} + \vec{S}$.\\

Mit den Beträgen der magnetischen Momente 
\begin{equation}
    |\vec{\mu_L}| = -\mu_B \sqrt{L(L+1)}
\end{equation}
\begin{equation}
    |\vec{\mu_S}| = - g_s \cdot \mu_B \sqrt{S(S+1)}
\end{equation}
lässt sich die Beziehung 
\begin{equation}
    |\vec{\mu_J}| = |\vec{\mu_S}| \cdot cos(\alpha) + |\vec{\mu_L}| \cdot cos(\beta)
\end{equation}
herleiten und zu 
\begin{equation}
    |\vec{\mu_J}| \approx \vec{\mu_B}\cdot g_J \sqrt{J(J+1)}
\end{equation}
approximieren.\\
Hier sind 