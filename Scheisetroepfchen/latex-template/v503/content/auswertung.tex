\section{Auswertung}
\label{sec:Auswertung}

%\begin{figure}
%  \centering
%  \includegraphics{plot.pdf}
%  \caption{Plot.}
%  \label{fig:plot}
%\end{figure}


Zuerst wird die Erfüllung der Bedingung $2 \cdot v_0 = v_{auf} - v_{ab}$ geprüft, um Werte, bei denen die Randbedingungen
des Experiments nicht erfüllt wurden, zu verwerfen. Es wird im folgenden nur mit den Messwerten weitergearbeitet, bei denen die 
prozentuale Abweichung vom Sollwert bei unter 50\% liegt. Sie sind in der \autoref{tab:geschw} gekennzeichnet. \\
Die Geschwindigkeiten in \autoref{tab:geschw} wurden via $v = \frac{s}{t}$ aus den gemessenen Zeiten errechnet.\\

\begin{table}[H]
  \centering
  \caption{Die Differenzen der Auf- und Abtriebsgeschwindigkeiten sowie die Ruhegeschwindigkeit der Öltröpfchen sowie ihre relative Abweichung vom Sollwert nach der Ruhegeschwindigkeit.}
  \begin{tabular}{ccccc}
    \toprule
    {$T  \mathbin{/} \; ° \unit{\celsius}$} &
    {$\mathbin{Spannung} \; \; U  \mathbin{/} \unit{\volt}$} &
    {$2 \cdot v_0 \mathbin{/} \mathbin{div / s}$} &
    {$v_{auf} - v_{ab} \mathbin{/} \mathbin{div / s}$} &
    {Abweichung $\mathbin{/} \unit{\percent}$}\\
    \midrule
21 & 4,57 &   -0,06  & 0,09  &  166,66 \\
21 & 4,57 &   -0,26  & 0,07  &  471,43 \\
21 & 4,57 &   -0,41  & 0,11  &  472,30 \\
21 & 4,57 &   0,12   & 0,21  &  \textbf{42,86}\\
21 & 4,57 &   -0,17  & 1,74  &  110,00\\
21 & 3,97 &   0,25   & 0,15  &  66,66\\
21 & 3,97 &   0,30   & 0     &  ungültig\\
21 & 3,97 &   -0,27  & 0,09  &  400,00\\
21 & 3,97 &   -0,17  & 0,13  &  230,77\\ 
21 & 3,97 &   -0,12  & 0,13  &  192,30\\
22 & 3,04 &   -0,24  & 0,04  &  700\\
22 & 3,04 &   0,13   & 0,23  &  \textbf{43,48}\\
22 & 3,04 &   0,26   & 0,14  &  85,71\\
22 & 3,04 &   0,32   & 0,13  &  146,15\\
22 & 3,04 &   -0,22  & 0     &  ungültig\\
23 & 4,91 &   0,65   & 0,09  &  622,22\\
23 & 4,91 &   0,29   & 0,10  &  190,00\\
23 & 4,91 &   -0,44  & 0,87  &  150,57\\
23 & 4,91 &   0,15   & 0,11  &  \textbf{36,36}\\
23 & 4,91 &   0      & 0,11  &  100,00\\
23 & 2,20 &   0,07   & 0,09  &  \textbf{22,22}\\
23 & 2,20 &   0,17   & 0     &  ungültig \\
23 & 2,20 &   -0,27  & 0,15  &  280,00\\
23 & 2,20 &   -0,23  & 0     &  ungültig\\
23 & 2,20 &   0,07   & 0,13  &  \textbf{44,00}\\

    \bottomrule
  \end{tabular}
  \label{tab:geschw}
\end{table}




\subsection*{Bestimmung der Elementarladung}

Für das elektrische Feld des Plattenkondensators wird die Formel
\begin{equation}
  E = \frac{U}{d}
\end{equation}
genutzt.\\
Es wird mit den Konstanten 
\begin{eqnarray}
  \rho_{Öl} = 886 \mathrm{\frac{kg}{m^3}}\\
  \rho_{Luft} = 1,16 \mathrm{\frac{kg}{m^3}}\\
  g = 9,81 \mathrm{\frac{m}{s}} \\
  d = (7,63 \pm 0,01) \mathrm{mm}
\end{eqnarray}
gerechnet, die bei Durchführung des Versuchs gegeben waren. \\

Nach \autoref{eq:radius} wird aus den berechneten Geschwindigkeiten der Radius
der Tröpchen bestimmt. Aus dem Radius wird durch \autoref{eq:ladung} die Ladung der Tröpchen berechnet und nach Cunningham durch \autoref{eq:cun} korrigiert. \\
So ergeben sich die Werte in \autoref{tab:ladung}, die in \autoref{fig:lad_unk} und \autoref{fig:lad_korr} graphisch dargestellt sind.\\

\begin{table}[H]
  \centering
  \caption{Die berechneten Radien, unkorrigierten und korrigierten Ladungen der Öltröpfchen.}
  \begin{tabular}{ccccccc}
    \toprule
    %{$T  \mathbin{/} \; ° \unit{\celsius}$} &
    {$U  \mathbin{/} \unit{\volt}$} &
   % {$2 \cdot v_0 \mathbin{/} \mathbin{div / s}$} &
    {$v_{auf} - v_{ab} \mathbin{/} \mathbin{div / s}$} &
    {Radius $r \mathbin{/} \unit{\metre} $} & %\cdot 10^{-7} doch hier oben rein?
    {unkorrigierte Ladung $q \mathbin{/} \unit{\coulomb}$} &
    {korrigierte Ladung $q \mathbin{/} \unit{\coulomb}$} \\
    \midrule
    4,57  & 0,21  &  1,17 $\cdot 10^{-06}$  &  (-1,43 $\pm$ 0,01) $\cdot 10^{-17}$  & (-1,59e-17 $\pm$ 0,01) $\cdot 10^{-17}$  \\
    3,04  & 0,23  &  1,43 $\cdot 10^{-06}$  &  (-2,75 $\pm$ 0,02) $\cdot 10^{-17}$ & (-2,98e-17 $\pm$ 0,02) $\cdot 10^{-17}$  \\
    4,91  & 0,11  &  2,14 $\cdot 10^{-06}$  &  (-3,03  $\pm$ 0,02) $\cdot 10^{-17}$   & (-3,21e-17  $\pm$ 0,02) $\cdot 10^{-17}$  \\
    2,20  & 0,09  &  1,76 $\cdot 10^{-06}$  &  (-2,49 $\pm$ 0,02) $\cdot 10^{-17}$  & (-2,66e-17  $\pm$ 0,02) $\cdot 10^{-17}$  \\
    2,20  & 0,13  &  0,91 $\cdot 10^{-06}$  &  (-1,41  $\pm$ 0,01) $\cdot 10^{-17}$    & (-1,60e-17  $\pm$ 0,01) $\cdot 10^{-17}$  \\
    \bottomrule
  \end{tabular}
  \label{tab:ladung}
\end{table}

Zur Berechnung der Elementarladung wird der euklidische Algorithmus verwendet.\\

\begin{figure}
  \centering
  \includegraphics{ladungen_unkorrigiert.pdf}
  \caption{Die Ladungen vor der Korrektur mit Fehlerbalken aufgetragen.}
  \label{fig:lad_unk}
\end{figure}

Durch die berechneten unkorrigierten Ladungen aus \autoref{fig:lad_unk} wird die experimentiell ermittelte Elementarladung $e_{0, \; unkorr} = -1,0 \cdot 10^{-19}$ C als der größte gemeinsame Teiler 
bestimmt. \\


\begin{figure}
  \centering
  \includegraphics{ladungen_neu.pdf}
  \caption{Die nach Cunningham korrigierten Ladungen mit Fehlerbalken aufgetragen.}
  \label{fig:lad_korr}
\end{figure}
Aus den zuvor nach Cunningham korrigierten Ladungen \autoref{fig:lad_korr} wird die experimentiell ermittelte Elementarladung zu einem Wert von $e_{0, \; korr} = -1,0 \cdot 10^{-19}$ C bestimmt. \\
Es wird analog zu $e_{0, \; unkorr}$ der größte gemeinsame Teiler berechnet.\\ 