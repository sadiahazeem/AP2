\section{Auswertung}
\label{sec:Auswertung}

%\begin{figure}
%  \centering
%  \includegraphics{plot.pdf}
%  \caption{Plot.}
%  \label{fig:plot}
%\end{figure}


Zuerst wird die Erfüllung der Bedingung $2 \cdot v_0 = v_{auf} - v_{ab}$ geprüft, um Werte, bei denen die Randbedingungen
des Experiments nicht erfüllt wurden, zu verwerfen. Es wird im folgenden nur mit den Messwerten weitergearbeitet, bei denen die 
prozentuale Abweichung vom Sollwert bei unter 50\% liegt. \\
Die Geschwindigkeiten in \autoref{tab:geschw} wurden via $v = \frac{s}{t}$ aus den gemessenen Zeiten errechnet.\\

\begin{table}[H]
  \centering
  \caption{Die Differenzen der Auf- und Abtriebsgeschwindigkeiten sowie die Ruhegeschwindigkeit der Öltröpfchen sowie ihre relative Abweichung vom Sollwert nach der Ruhegeschwindigkeit.}
  \begin{tabular}{ccccc}
    \toprule
    {$T  \mathbin{/} \; ° \unit{\celsius}$} &
    {$\mathbin{Spannung} \; \; U  \mathbin{/} \unit{\volt}$} &
    {$2 \cdot v_0 \mathbin{/} \mathbin{div / s}$} &
    {$v_{auf} - v_{ab} \mathbin{/} \mathbin{div / s}$} &
    {Abweichung $\mathbin{/} \unit{\percent}$}\\
    \midrule
21 & 4,57 &   -0,06  & 0,09  &  166,66 \\
21 & 4,57 &   -0,26  & 0,07  &  471,43 \\
21 & 4,57 &   -0,41  & 0,11  &  472,30 \\
21 & 4,57 &   0,12   & 0,21  &  42,86\\
21 & 4,57 &   -0,17  & 1,74  &  110,00\\
21 & 3,97 &   0,25   & 0,15  &  66,66\\
21 & 3,97 &   0,30   & 0     &  ungültig\\
21 & 3,97 &   -0,27  & 0,09  &  400,00\\
21 & 3,97 &   -0,17  & 0,13  &  230,77\\ 
21 & 3,97 &   -0,12  & 0,13  &  192,30\\
22 & 3,04 &   -0,24  & 0,04  &  700\\
22 & 3,04 &   0,13   & 0,23  &  43,48\\
22 & 3,04 &   0,26   & 0,14  &  85,71\\
22 & 3,04 &   0,32   & 0,13  &  146,15\\
22 & 3,04 &   -0,22  & 0     &  ungültig\\
23 & 4,91 &   0,65   & 0,09  &  622,22\\
23 & 4,91 &   0,29   & 0,10  &  190,00\\
23 & 4,91 &   -0,44  & 0,87  &  150,57\\
23 & 4,91 &   0,15   & 0,11  &  36,36\\
23 & 4,91 &   0      & 0,11  &  100,00\\
23 & 2,20 &   0,07   & 0,09  &  22,22\\
23 & 2,20 &   0,17   & 0     &  ungültig \\
23 & 2,20 &   -0,27  & 0,15  &  280,00\\
23 & 2,20 &   -0,23  & 0     &  ungültig\\
23 & 2,20 &   0,07   & 0,13  &  44,00\\

    \bottomrule
  \end{tabular}
  \label{tab:geschw}
\end{table}




\subsection{Bestimmung der Elementarladung}

Für das elektrische Feld des Plattenkondensators wird die Formel
\begin{equation}
  E = \frac{U}{d}
\end{equation}
genutzt.\\
Es wird mit den Konstanten 
\begin{eqnarray}
  \rho_{Öl} = 886 \mathrm{\frac{kg}{m^3}}\\
  \rho_{Luft} = 1,16 \mathrm{\frac{kg}{m^3}}\\
  g = 9,81 \mathrm{\frac{m}{s}} \\
  d = (7,6250 \pm 0,0051) \mathrm{mm}
\end{eqnarray}
gerechnet, die bei Durchführung des Versuchs gegeben waren. \\

Nach xxxxx wird aus den berechneten Geschwindigkeiten der Radius
der Tröpchen bestimmt. Es ergeben sich die Werte in xxxx.

\begin{table}[H]
  \centering
  \caption{Die Differenzen der Auf- und Abtriebsgeschwindigkeiten sowie die Ruhegeschwindigkeit der Öltröpfchen sowie ihre relative Abweichung vom Sollwert nach der Ruhegeschwindigkeit.}
  \begin{tabular}{ccccc}
    \toprule
    {$T  \mathbin{/} \; ° \unit{\celsius}$} &
    {$\mathbin{Spannung} \; \; U  \mathbin{/} \unit{\volt}$} &
    {$2 \cdot v_0 \mathbin{/} \mathbin{div / s}$} &
    {$v_{auf} - v_{ab} \mathbin{/} \mathbin{div / s}$} &
    {Radius $\mathbin{/} \unit{\metre}$}\\
    \midrule
    21 & 4,57 &   0,12   & 0,21  &  42,86\\
    22 & 3,04 &   0,13   & 0,23  &  43,48\\
    23 & 4,91 &   0,15   & 0,11  &  36,36\\
    23 & 2,20 &   0,07   & 0,09  &  22,22\\
    23 & 2,20 &   0,07   & 0,13  &  44,00\\


    \bottomrule
  \end{tabular}
  \label{tab:geschw}
\end{table}
