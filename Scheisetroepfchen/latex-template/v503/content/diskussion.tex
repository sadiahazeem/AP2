\section{Diskussion}
\label{sec:Diskussion}


Bei der Durchführung der Messung fällt auf, dass die 
Geschwindigkeitsbestimung der Öltröpfchen sehr zeitintensiv ist, da während der 
Durchführung sowie bei Auswertung der Daten viele Werte verworfen werden müssen. \\
Im Vergleich mit dem Literaturwert von $e_{0, \; Lit} = 1,602 \cdot 10^{-19}$ C \cite{enull} zeigt sich, dass der korrigierte Wert für $e_0$ eine 
relative Abweichung von $\eta_{e_0} = 6,3 \%$ hat, der unkorrigierte von $\eta_{e_0} = 13,2 \%$. \\
Beide Abweichungen liegen im tolerablen Bereich. Es zeigt sich außerdem, dass die Cunningham-Korrektur
eine höhere Genauigkeit gegenüber unkorrigierten Werten hat.\\
Nichtsdestotrotz bietet die Aufnahme der Messwerte durch simultanes Beobachten der Öltröpfchen, Stoppen der Zeit und Steuern des 
elektrischen Feldes viel Spielraum für systematische Messfehler. \\