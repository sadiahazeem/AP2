\section{Diskussion}
\label{sec:Diskussion}


Bei der Durchführung der Messung fällt auf, dass die 
Geschwindigkeitsbestimung der Öltröpfchen sehr zeitintensiv und 
ungenau ist, da während der Durchführung sowie bei Auswertung der Daten viele Werte verworfen werden müssen. \\
Im Vergleich mit dem Literaturwert zeigt sich, dass beide experimentiell bestimmten Werte für $e_0$ eine 
relative Abweichung von $\eta = 162,42 \%$ haben. \\
Diese Ungenauigkeit könnte darauf zurückgeführt werden, dass je Tröpfchen nicht mehrere Messreihen durchgeführt werden 
konnten. \\ 
Außerdem bietet die Aufnahme der Messwerte durch simultanes Beobachten der Öltröpfchen, Stoppen der Zeit und Steuern des 
elektrischen Feldes viel Spielraum für systematische Messfehler. \\