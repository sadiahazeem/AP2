\section{Durchführung}
\label{sec:Durchführung}
Zur Verfügung stehen zwei 2MHz-Ultraschallsonden, ein Ultraschallechoskop,
ein Rechner zur Datenaufnahme und -analyse, als Kontaktmittel bidestilliertes Wasser sowie Ultraschallgel
und als Probenstücke sechs Acrylformen sowie ein Augenmodell.

\subsection{Bestimmung der Schallgeschwindigkeit}
Zuerst werden mit der Schieblehre alle Acrylformen ausgemessen. 
\subsubsection*{Impuls-Echo-Verfahren}
Als Kontaktmittel wird zuerst bidestilliertes Wasser aufgetragen.
Mit einem A-Scan wird für acht verschiedene Längen der Acrylzylinder die Laufzeit des 
Ultraschallsignals aufgenommen.


\subsubsection*{Durchschallungsverfahren}
Beim Durchschallungsverfahren werden die Acrylzylinder in eine horizontale Halterung gelegt.
Als Kontaktmittel wird Ultraschallgel verwendet und die Sonden werden von 
beiden Seiten an die runden Grundflächen der Zylinder gekoppelt.
Erneut werden die Laufzeiten der Signale aufgenommen. 


\subsection{Bestimmung eines Dämpfungskoeffizienten}
\subsubsection*{Impuls-Echo-Verfahren}
Es werden für sieben verschiedene Längen der Acrylzylinder die 
Amplituden des ausgesendeten und reflektierten Ultraschallsignals gemessen. 


\subsection{Biometrische Untersuchung eines Augenmodells}
Die Abstände im Augenmodell sollen mit dem Impuls-Echo-Verfahren ausgemessen werden. Zu diesem Zweck wird 
eine Sonde auf die Hornhaut des Modells gesetzt und alle reflektierten Signale des A-Scans aufgenommen. 