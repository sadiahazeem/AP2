\section{Diskussion}
\label{sec:Diskussion}


Die Abweichung der experimentiell bestimmten Schallgeschwindigkeit in Acryl von $c = (2714 \pm 30) \mathrm{\frac{m}{s}}$ weicht vom Literaturwert von $c = 2730 \mathrm{\frac{m}{s}}$ um
\begin{equation*}
\eta_{Schallgeschwindigkeit} = (0,6 \pm 1,1) \%
\end{equation*}

ab. Mögliche Fehlerquellen sind hier vor allem in der Aufnahme der Messwerte zu finden. 
Der Umgang mit den Ultraschallsonden erfordert etwas Erfahrung, die Messreihe zur experimentiellen Bestimmung der Schallgeschwindigkeit
war jedoch die erste durchgeführte Arbeit mit dem Messgerät.\\

Im Vergleich zwischen A- und B-Scan zeigen sich deutlich höhere Abweichungen, insbesondere bei der Bestimmung der Durchmesser der 
Störstellen. Beispielsweise weicht der im A-Scan bestimmte Durchmesser der Störstelle 5 mit $d_A = 3,4$ mm um 
\begin{equation*}
  \eta_A = 21,8 \%
\end{equation*}

von dem mit der 
Schieblehre ausgemessenen Wert $d_{Schieblehre} = 2,8$ mm ab, im B-Scan wurde derselbe Durchmesser jedoch zu $d_B = 2,1$ mm bestimmt. $d_B$ weicht damit 
um 
\begin{equation*}
  \eta_B = 26,1 \%
\end{equation*}

von $d_{Schieblehre}$ ab.\\
Die Abweichungen in der Lagebestimmung drei zufällig ausgewählter Störstellen sind in \autoref{tab:abw} aufgelistet
\begin{table}[H]
    \centering
    \caption{Die errechneten Tiefen $s$ der Störstellen und ihre Abweichungen $\eta$, von unten und von oben. Alle Strecken in mm und Abweichungen in Prozent.}
    \begin{tabular}{ccccccccccc}
      \toprule
      {Lochnummer} & 
      {$s_{u}$} &
      {$s_{o}$} &
      {$s_{u, \; A}$} &
      {$\eta_{u, \; A}$} &
      {$s_{o, \; A}$} &      
      {$\eta_{o, \; A}$} &
      {$s_{u, \; B}$} &
      {$\eta_{u, \; B}$} &
      {$s_{o, \; B}$} &      
      {$\eta_{o, \; B}$} \\
      \midrule
      %     unten  oben     a unten          a oben          b unten          b oben      
      3  &  23,4  & 55,1  &  24,2   &  3,4  & 55,6  &  0,9  &  24,8 &  6,2   &  48,3 &  12,3  \\
      7  &  54,2  & 22,1  &  54,9   &  1,3  & 22,9  &  3,6  &  51,1 &  5,7   &  16,6 &  24,9  \\
      G  &  16,4  & 55,5  &  16,7   &  1,8  & 56,5  &  1,8  &  15,2 &  7,3   &  56,6 &   2,0  \\
      \bottomrule
    \end{tabular}
 \label{tab:abw}
\end{table}
\noindent
Es lässt sich somit schlussfolgern, dass der A-Scan sowohl geeigneter zur Ausmessung der Größe einer Störstelle, als auch 
präziser in der Lokalisation ist. 
Außerdem fällt auf, dass die Abweichungen im von oben aufgenommenen B-Scan zum Großteil deutlich höher sind, als in dem, der von unten aufgenommen wurde.
Dies kann darauf hinweisen, dass im Scan die Ultraschallsonde zu schnell über den Block bewegt wurde oder ein anderer systematischer 
Fehler unterlaufen sein könnte.\\ 

Die Tumore im Brustmodell werden im B-Scan in einer realistischen Dicke dargestellt.
Es lässt sich im B-Scan außerdem problemlos der Aggregatszustand des Inhalts der Tumore in fest und flüssig unterscheiden. 